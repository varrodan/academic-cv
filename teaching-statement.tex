\documentclass[a4paper,11pt]{article}
%\documentclass[a4paper,12pt]{article}
%\usepackage[backend=biber,sorting=ydnt,maxnames=18,giveninits=true,labelnumber=true,defernumbers=true]{biblatex}
%\usepackage[backend=biber,sorting=ydnt,maxnames=18,giveninits=true]{biblatex}
%\usepackage[bibstyle=publist]{biblatex}
\usepackage{xstring}
\usepackage{url}
\usepackage{booktabs}
\usepackage{hyperref}

\usepackage{titlesec}
\titlespacing*{\section}{0pt}{12pt plus 1pt minus 0.5pt}{6pt plus 0.5pt}
%\titlespacing*{\subsection}{0pt}{5.5ex plus 1ex minus .2ex}{4.3ex plus .2ex}

\titlespacing*{\paragraph}
{0pt}{3pt plus 1pt minus 0.5pt}{3pt plus 2pt minus 0.5pt}

\usepackage{enumitem}
\setlist{nosep}

\usepackage{tcolorbox}

\usepackage{fontspec}
\setmainfont{Cambria}
\setsansfont{Calibri}

%\setlength{\itemsep}{0pt plus 1pt}
%\setlength{\itemsep}{0pt}
%\setlength{\topsep}{0pt}
\setlength{\parskip}{3pt plus 0.5pt minus 0.5pt}
%\setlength{\beforeskip}{3pt}


%\usepackage[scale=0.85]{geometry}
\usepackage[scale=0.85]{geometry}

\geometry{
 a4paper,
 total={170mm,257mm},
 left=20mm,
 top=20mm,
 }

%\assignrefcontextkeyws[labelprefix=C]{C}
%\assignrefcontextkeyws[labelprefix=J]{J}

\usepackage{fancyhdr}
\pagestyle{fancy}\lhead{Teaching Statement} \rhead{May 2019}
\chead{{\bf Dániel Varró}} \lfoot{} \rfoot{\bf \thepage} \cfoot{}

\title{Teaching Statement}
\author{D\'aniel Varr\'o}
\date{May 2019}

\begin{document}
\maketitle
\setcounter{page}{17}

\vspace{12pt}

%\section{Summary of Excellence}
\begin{tcolorbox}[title=Teaching Philosophy]
As a \emph{course lecturer}, a main philosophy of mine is that a software engineering related course should \emph{simultaneously teach foundational concepts and cutting edge technology}. Without the former, a course becomes empty - without the latter, a course becomes irrelevant. I am dedicated to bring \emph{realistic engineering challenges} (frequently with real customers) to a course that \emph{facilitate collaboration} within and between student teams while they acquire the new concepts and technologies. I regularly use various  means of \emph{active learning} to make my lectures lively. Furthermore, I always wish to maintain an \emph{inclusive atmosphere} even for a very \emph{diverse and international audience}. 
\vspace{3pt}

As a \emph{supervisor} who actively collaborates with his graduate students with in-depth discussions, my goal is to continuously \emph{strive for excellence} both in engineering tasks as well as in publications while my graduate students are gradually becoming more and more \emph{independent researchers with critical thinking}.  My students typically undertake deep research tasks which go well beyond the state-of-the-art. As a further characteristic, many of our past \emph{results are simultaneously precise and scalable}, and it has been actively \emph{used in an industrial setting}. Active investment in research prototype tools help my students become \emph{excellent software engineers and architects}.
\end{tcolorbox}


\section{Teaching experience and effectiveness at different levels}
Starting as a teaching assistant in 1999, I have been involved in teaching at various levels as a full-time staff member at the Budapest University of Technology and Economics (BME, lecturing dominantly in Hungarian) between 2003 and 2016 and since then at McGill University (Canada, lecturing in English). The list of all courses I taught are listed in my resume.

\paragraph{Undergraduate-level courses.}
At \emph{BME}, I was involved in teaching a course on \emph{Formal methods} (with 300+ students each year) since the beginning of the course in 2001 until 2006. Moreover, I was a key contributor and strategic founder of a \emph{curriculum on Systems Engineering}. In a supervisor role, I participated in introducing virtual labs over educational cloud (using Apache VCL) at our faculty for the first time in Hungary which received a \href{https://inf.mit.bme.hu/en/news/2014/02/tempus-stem-call-apache-vcl-based-labs-won-prize}{TEMPUS STEM Award} in 2014. 

At \emph{McGill University}, I was the instructor of a course on \emph{Introduction to Software Engineering} for three times with 75-95 students in 2017-2019 and for a very diverse and international audience. In all three editions, I received an average score between 4.2-4.5 (out of 5.0) for all course evaluation questions related to me as instructor, which was significantly over the department average (by an average of 0.6). In 2018, I significantly modernized the technological background of the course – which now teaches modern software engineering principles and architectures in the context of modern software technologies. Despite the heavy workload, the vast majority of students loved the course and gave very positive feedback, such as “\emph{It has been the most interesting and educational course I have taken. So much effort is put in this course from the detailed \href{https://mcgill-ecse321-winter2019.github.io/EventRegistration-Tutorials/}{Hands-On tutorial} document, to the project requirements supported by ‘real’ clients, and finally, the \href{https://flipquiz.com/flashcards/82085/}{interesting games}!}”, “\emph{Professor Varro is by far one of the most helpful, dedicated professors at McGill}”. At McGill, I voluntarily took several trainings offered by Teaching and Learning Services at McGill on how to increase student participation by \emph{active learning} (e.g. frequent online polls during lectures). In Fall 2018, I also taught a course on \emph{Software validation} for over 120 students, but future editions of this course will need significant further modernization.

\paragraph{Graduate-level courses.}
At BME, I developed a course on \emph{UML-based modeling and analysis} which ran between 2003 and 2009 for 70-80 MSc students. \emph{Several teaching assistants of the course were former students} of the course who \emph{voluntarily came back from industry to help the course}, which was a highly unusual way of teaching in Hungary. Between 2009-2016, I ran a course on “Model-driven software/systems development” (twice in English) with 15-20 students each year. I also actively contributed to start the first ever Hungarian university course on \emph{Eclipse based technologies} in 2005. I was also involved in several industrial trainings on these topics. I have been running a PhD seminar on the \emph{Foundations of Model-Driven Engineering} between 2004 and 2014 (with 5-8 students each year). At McGill University, I offer a graduate course on 
%\emph{model-based engineering for cyber-physical systems} 
\emph{Critical systems} 
which teaches SysML and related standards in the context of Internet-of-Things and safety-critical CPS.


\section{Talent care and supervision of graduate students}
\paragraph{Supervising PhD and MSc theses.}
Starting from my early PhD studies, I have been active in talent care by supervising MSc and PhD students. I have been the \emph{main scientific advisor of 11 PhD students} (5 completed, 3 PhD candidates to complete in Summer 2019), and co-supervised 3 more PhD students (2 completed, 1 PhD candidate). I also supervised 20+ MSc theses since 2001. All of my \emph{seven best paper awards had one of my PhD students} as the first author. Moreover, my PhD students were successfully competing at prestigious international Student Research Contests and Doctoral Symposiums (2x 1st prize at MODELS, 1x 2nd prize at SIGMOD, 1x 1st prize at STAF conferences).

\paragraph{Supervising early research work.} 
23 scientific reports written by MSc students under my tutoring participated successfully in a national two-phase (faculty and national-level) \href{http://www.otdt.hu/hu/cms/otdk/orszagos-tudomanyos-diakkori-konferencia/}{Scientific Students’ Associations Conference} organized for students who carry out early research work in Hungary. Early research results achieved by MSc students provided a basis for over 25 scientific publications published at international conferences. In 2009, I was selected as a \href{http://www.otdt.hu/page/kituntetesek/mak2009.php}{Distinguished Tutor} by the National Scientific Students Council (OTDT), a national prize requiring 10 years of successful tutoring, awarded bi-annually to only 3 tutors in the field of computer science. This way, I started to focus on talent care very early, in fact, as a 1st year PhD student. I was the \emph{first ever recipient} of the \href{https://otdk2017.mik.uni-pannon.hu/index.php/eredmenyek}{Csanád Imreh Award}, commemorating the Hungarian researcher who died %tragically 
at a young age. 

\paragraph{Careers of former students.} Together with 4 of my former PhD students (István Ráth, Ákos Horváth, Gábor Bergmann and Ábel Hegedüs), we \emph{founded a start-up company, IncQuery Labs Ltd.} in 2013 to provide industrial exploitation of our research results and our expertise in model-driven engineering. András Balogh (whom I co-supervised) is now Chief Technology Officer at ThyssenKrupp Presta Hungary. Four of my former MSc students worked for Google, several of them went for \emph{large international companies} Nokia, Ericsson, Lufthansa Systems, National Instruments or small \emph{innovative startups}.

\section{Experience in educational leadership}
%(on research group, department, faculty, national level) 
At BME, I was the \emph{operative leader} of the \href{http://inf.mit.bme.hu/en/}{Fault Tolerant Systems Research Group} (consisting of over 20-25 members) between 2012 and 2016. Within this role, I was in charge of coordinating many aspects of the everyday life and duties of the group including the strategic supervision of educational activities. 

Since 2012, I am also a member of the \emph{strategic executive board of the department}. In 2013, I was nominated as a \emph{department representative at faculty-level coordination meetings} aiming for the development of a new curriculum on the MSc level, which officially started in February 2015. I have continuously been serving in \emph{cross-department committees for talent care} in the past 5 years. 

On the faculty-level, I am member of the Informatics Doctoral School at the Faculty of Electric Engineering. Between 2014-16, I have been serving on the committee of the Doctoral School on Software Engineering (Informatics) to shape the education program of PhD students. On the national-level, I am a \emph{member of the Informatics Scientific Committee of the Hungarian Academy of Sciences} (elected as the youngest member first in 2011 and re-elected in 2014 and 2017). I was an external member of the Informatics Doctoral Committee of University of Szeged.

At McGill University, I served on \emph{various senior committees on the department level} including faculty search, graduate student ranking for scholarships, strategic advisory board of the department chair, tenure and reappointment committee. On the university-level, I have been elected to serve as one of the two the \emph{representatives of Faculty of Engineering at the Council of Graduate and Postdoctoral Studies}. I also serve as the \emph{program director of the upcoming software engineering co-op program } where undergraduate students will need to spend four compulsory internship terms at companies.

\section{Teaching and graduate supervision philosophy}

Most of my courses have been organized along a common scheme. Students need to invest significant amount of work in completing a \emph{team project} of 3-5 persons. I assign a \emph{complex software/systems engineering challenge}, and ideally with a real customer, where students need to \emph{use modern software technologies} (e.g. IDEs, version control systems, etc.) and project management frameworks (like Basecamp, Trello) while learning also foundational concepts. 

Maintaining the attention of students during lectures can be a challenging task. I try to postpone abstract mathematical definitions to the point when the main concepts are sufficiently demonstrated by small examples first. I frequently use various \emph{best practices of active learning} during my lectures (e.g. bug hunts on ill-formed software, various games and quizzes, small challenges discussed in small groups). I often \emph{demonstrate the practical relevance of concepts using real-world examples} and industrial case studies sometimes presented by industrial collaborators within invited lectures. Usually, \emph{my lectures are complemented with tutorials} on key technologies (e.g. Eclipse technologies or multi-tier web applications) actively co-developed by my graduate students which also provides a first teaching experience for them. 

\paragraph{Supervision.} 
I feel strong at \emph{motivating students to start exploring highly uncharted research areas}. The lack of significant previous work in a certain area frequently boosts the motivation of talented students. When publishing results, I help identify typical errors committed by students due to their lack of expertise in scientific writing which helps them achieve early successes. Then I gradually let them act more and more on their own to gradually gain independence in writing by the time they defend their PhD. Moreover, \emph{several PhD projects were grouped around complex open source projects and demonstrators} which facilitate team work between students and provide a greater visibility of their research for the public. My students have also frequently participated in long-term industrial collaborations in research intensive tasks.


%\begin{figure}
%\centering
%\includegraphics[width=.8\textwidth]{figures/teaching-eval}
%\caption{Teaching evaluation at McGill Univeristy}
%\label{fig:teaching-eval}
%\end{figure}



\end{document}
