\chapter{Service Portfolio}
\label{sec:service-portfolio}
\lfoot{Service Portfolio} 

\section{Service for Scientific and Professional Communities}

%\subsection{Summary of service for scientific communities (entire career)}


\subsection{Conference \& workshop organization, Editorial boards}

As a reflection of my reputation in my research area, I served in various key organizational roles for leading scientific conferences and journals after joining McGill University in 2016. 

\begin{itemize}[leftmargin=0.5cm]
\item \textbf{Program co-chair}: Together with Emilie Balland, we served as the program co-chairs of the \emph{9th ACM SIGPLAN International Conference on Software Language Engineering} (\href{https://www.sleconf.org/2016/}{SLE 2016}) hosted in Amsterdam, Netherlands. The SLE conference is devoted to the principles of software languages: their design, their implementation, and their evolution. As a PC co-chair, I was in charge of preparing the call for papers of the conference, organizing the entire review process to ensure the fair evaluation of papers, making decisions on the final selection of papers included in the conference program, assembling the different sessions of the conference, inviting session chairs, etc. 

\item \textbf{Program co-chair}: Together with Marsha Chechik and Daniel Strüber, we served as the program co-chairs of the \emph{11th Workshop on Modelling in Software Engineering} (\href{https://sselab.de/lab2/public/wiki/MiSE/index.php}{MiSE’2019}) hosted by ICSE 2019 in Montreal, Canada. This 2-day workshop is a traditional satellited event of the ICSE conference, and it is regarded almost as a conference by the software and systems modeling community. My duties in the organization included to prepare the workshop proposal (as even workshop selection is competitive), organizing the review process for the conference, etc.

\item \textbf{Program board member}: I served on the program board (consisting of 13 senior researchers of the community to overview the review process) in 2016 and 2017 for the \emph{IEEE / ACM International Conference on Model Driven Engineering Languages and Systems} (\href{MODELS}{http://modelsconference.org/}), which is the main scientific forum of my research area. In this role, I was coordinating the review process of papers assigned to me and helped make final decisions on those papers.

\item \textbf{Posters co-chair}: Together with Wahab Hamou-Lhadj, I served as a co-chair of the \emph{Posters Track of the 41st International Conference on Software Engineering} 
(\href{https://2019.icse-conferences.org/track/icse-2019-Posters}{ICSE 2019}), which is the flagship conference on software engineering. In this role, I was contributing to organizing the review process of poster submissions, collaborating with co-chairs of other ICSE tracks, establishing a selection committee, etc.

\item \textbf{Steering committee member}: I continued to serve on the steering committee (SC) of the \emph{International Conference on Model Transformation} (\href{http://www.model-transformation.org/}{ICMT}), which is the major topical conference of my direct research area. My duties included the strategic planning of future editions of the conference (e.g. selecting program chairs and locations) and providing further strategic guidance.

\item \textbf{Editorial board member}: I continued to serve on the editorial board of the \emph{Software and Systems Modeling} (\href{http://sosym.org/}{SoSyM}) journal (Springer), which is the main scientific journal for my research area of model-based software and systems engineering. In this role, I am in charge of organizing the review of 1-3 papers each year by selecting qualified reviewers and making recommendations on acceptance or rejection based upon reviewers' feedback.

\item \textbf{Editorial board member}: I was invited to joined the editorial board of \emph{Journal of Object Technology} (\href{http://www.jot.fm/}{JOT}),  a peer-reviewed, free and open-access journal included by all major indexing services. 
\end{itemize}

\paragraph{Entire career}
During my entire career, I served practically in all major roles in the organization of scientific conferences. Here I provide a 1-paragraph brief summary of the highlights of my academic roles. I was \emph{program co-chair} of three major software engineering conferences (FASE 2013, ICMT 2014 and SLE 2016) as well as numerous workshops. I was the \emph{general chair} of the first ever STAF conference (Software Technologies: Applications and Foundations) in 2013, the Fourth Int. Symposium on Applications of Graph Transformations with Industrial Relevance (AGTIVE 2011) and the SENSUS 2009 international summer school. I was \emph{workshops co-chair} at STAF 2016, \emph{doctoral symposium co-chair} at STAF 2015.
At the IEEE / ACM Int. Conf. on Model Driven Engineering Languages and Systems (MODELS), which is the main scientific venue of my research field, I was \emph{demos and exhibitions chair} at MODELS 2012, academic posters and demos chair at MODELS 2007.  I was the \emph{local organizing chair} of European Joint Conferences on Theory and Practice of Software (ETAPS 2008) with 670 participants and \emph{served on the ETAPS steering committee} for a total of six years. 
I served on the \emph{editorial board of the SoSyM journal}, the main scientific journal of my research area since 2011.  

\subsection{Program committees and reviewing activities}
Since joining McGill University in 2016, I have been continuously received invitations to serve as a reviewer in different program committees (PCs), project evaluation boards or in journals, especially, in the field of software engineering and formal methods. 

\begin{itemize}[leftmargin=0.5cm]
\item \textbf{PC member at ICSE:}
For the first time in my careers, I was invited to \emph{serve in the program committees of ICSE 2018 and ICSE 2019}, the IEEE/ACM Int. Conf. on Software Engineering, which is the main scientific forum of software engineering. My thorough reviewing activities were recognized by the highly prestigious \textbf{ACM Distinguished Reviewer Award} for ICSE 2018 (only 11 awardees out of 101 PC members). 

\item \textbf{PC member at MODELS:}
For the IEEE / ACM Int. Conf. on Model Driven Engineering Languages and Systems (MODELS), which is the main scientific forum of my research area, I served on the Program Board (consisting of 13 senior researchers of the community to overview the review process and make final decisions on papers) in 2016 and in 2017 and on the program committee in 2018 and 2019. 

\item \textbf{PC member at other major conferences:}
I also served on the program committee of other major software engineering conferences including the 
Int. Conf. on Model Transformation (ICMT 2017 and 2018), the European Conference on Modelling Foundations and Applications (ECMFA 2018 and 2019), the Tools Track of the IEEE Automated Software Engineering Conference (ASE 2016), Fundamental Approaches to Software Engineering (FASE 2017 and 2018). I also got PC invitations for conferences in the area of \emph{formal methods}.

\item \textbf{Project proposal reviewer:}
I served as an external evaluator of various project proposals including 1x VICI grant from Netherlands, 1x EU COST Action, 1x NSERC Discovery Grant, 16 Bolyai Scholarship applications in Hungary, 1x Women for Science proposal for the Hungarian Academy of Sciences. I also had to decline 3 further review invitations due to conflict of interest.

\item \textbf{Journal reviewer:}
I regularly reviewed for major scientific journals of my research area (software engineering, formal methods) including 4 papers at IEEE Transactions on Software Engineering, 2x for Formal Aspects of Computing, 2x for Software and Systems Modeling, 1x for IEEE Software). Please also note that it is my regular practice to "constructively decline" journal review invitations and recommend my former graduate students as qualified reviewers for the same paper. I strongly believe that this practice helps their integration to the scientific community. 
\end{itemize}

\paragraph{Entire career}
I served on the program committee of leading international conferences and workshops in different research fields: 66 times in software engineering, 16x at visual modeling techniques and tools, 14x at formal methods, 4x at depedendable computing. I served on the senior program board of the MODELS conference twice. I acted as an external evaluator of project proposals in 7 different countries and an external reviewer of tenure / promotion dossiers submitted in UK, Canada and South Africa. 

\subsection{Standardization activities}
In December 2017, the Object Management Group (OMG) issued a Request for Proposal (RFP) for SysML V2, the next-generation systems modeling language standard. In 2018, I regularly participated in the standardization activities of the so-called SST submission group, which is a very large group of companies and universities collaborating to define the next generation of the standard (including e.g. Boeing, Airbus, NASA Jet Propulsion Lab, Siemens, etc.).

As a representative of IncQuery Labs Ltd. (the start-up company I co-founded in Hungary), I partcipated in three in-person meetings of the standardization group (in Boston, June 2018, in Ottawa, September 2018 and in Seattle in December 2018) and several teleconferences. My specific role was related to define a query language and an open application programming interface (API) for SysML models. In particular, the envisioned query language should be build on the VIATRA Query Language, which an the open source graph query language based on past research led by me. While the official submission deadline of the related standard is only in Spring 2020, even the potential transition of our query language to the systems engineering standard modeling language is major impact of our past research activities. 

\section{University Service}

\subsection{University-related review activities}
I was very active in university-related review activities both at McGill University and on an international level. Within McGill, I served as the \emph{internal examiner} of 2 PhD theses, a \emph{member of the oral PhD committee} once, and acting as Pro-Dean once. In addition, I was the sole \emph{thesis examiner} of 4 MEng/MSc thesis at McGill. In addition, I served as a \emph{member the qualifying exam committee} for 2 PhD students at McGill. 

On an international level, I was invited to serve as an \emph{external reviewer} of several PhD theses, once in Austria, once in Germany and once in UK. In addition, I served as an examiner and the chair of the complex exam committee at BME in Hungary. 

\paragraph{Entire career}
During my entire career, I was inivited to serve as an expert (i.e. reviewer or committee member) in a total of 24 PhD defenses (7 of which since joining McGill) in 9 different countries while acting as the chair of the PhD defense committee once. 

\subsection{Departmental and faculty committees}
During the 3 years working at McGill University, I served on 8 different committees for the ECE Department, and 1 working group for the Faculty of Engineering, while acting as an elected representative of the Faculty of Engineering for the Council of Graduate and Postgraduate Studies (CGPS). I served as a member of the Search Committee of the ECE Department for two consecutive hiring rounds (2018 and 2019) evaluating application packages of hundreds of applicants. 
%I serve as the program director of the upcoming software engineering co-op program, which is further detailed as part of my Teaching Portfolio.

%\begin{description}
\begin{yearlist}
\item[2018-19] \textbf{Program director for software engineering co-op}: Reported in Teaching Portfolio
\item[2017-19] \textbf{Faculty of Engineering representative of the Council of Graduate and Postdoctoral Studies (CGPS)}: The mandate of the university-level CGPS council is to evaluate proposal made to different graduate-level programs proposed by the different faculties. As an elected representative of the Faculty of Engineering, I am in charge of assessing and voting about such proposals. In addition, I also need to report about engineering-related discussions and changes in the graduate programs at the Faculty Council meetings. 
\item[2018-19] \textbf{Curriculum Committee member, ECE}: 
This committee is responsible for the maintenance and renewal of the department’s undergraduate programs in Electrical Engineering, Computer Engineering, and Software Engineering. My main contributions to the committee are related to the introducing specialization streams (i.e. consistent groups of elective courses taken by students) for the Software Engineering program. I also proposed changes to the sample Software Engineering curriculum of the upcoming co-op program, especially, with respect to the scheduling of the compulsory co-op terms. Since the SE coop program did not finally start in Fall 2019, the final approval of these proposals were postponed to Fall 2019.

\item[2018-19] \textbf{Undergraduate Advising Committee member, ECE}: 
The mandate of this committee is to provide advising for undergraduate students about anything related to their studies. 
In this role, I provided advise for undergraduate students on course selection, program planning, resolving their scheduling conflicts, and other related issues intensively at the first three weeks of each semester, and on an appointment basis afterwards. 


\item[2018-19] \textbf{Student Exchanges \& Study Abroad Committee member, Faculty of Engineering}: I am the ECE representative of this faculty-level committee which promotes and coordinates various international exchange opportunities activities for undergraduate and graduate students. During the year, I was supporting the committee's decision to increase its mandate to other forms of experiential learning. 
\item[2017-19] \textbf{Departmental Search Committee member, ECE}: The mandate of this committee is to make recommendations for the department chair on which candidate to hire for an open professor position. My duties included to attract strong software engineering candidates (e.g. personal contacts or on mailing lists), to rank hundreds of application packages, to discuss applications with other committee members, to assess long-listed candidates via dozens of Skype interviews, to attend several hiring seminars, and to interact with the candidates in person to know their deeper personal and social context.
\item[2017-19] \textbf{GitHub Enterprise Working Group member, Faculty of Engineering}: This working group was in charge of running a pilot project on evaluating the use of the enterprise version of the popular GitHub version control system hosted locally at McGill premises. My main contribution was to voluntarily use the software for the software team projects as part of the ECSE 321 Introduction to Software Engineering course, and provide feedback about its use during the monthly meetings of the working group. A key insight provided by me was that this academic version of GitHub Enterprise is not compatible with academic licenses of other cloud-based software engineering products used for continuous integration (e.g. Travis CI).
\item[2016-18] \textbf{Chairman's Advisory Committee member}: This committee is composed of senior professors of the ECE department, and its mandate is to provide strategic advice for the department chair. My contribution was to highlight strategic areas in software engineering which are not sufficiently covered by existing professors, such as software security. These areas were used by the Departmental Search Committee when evaluating candidates for academic positions at subsequent years.
\item[2017-18] \textbf{Promotions and Reappointments Committee member}: This committee evaluates the compulsory third year report of tenure-track professors and makes suggestions for the feedback provided by the department chair. I was involved in reading and evaluating the report of one tenure-track professor of the ECE department. 
\item[2016-17] \textbf{Departmental Tenure Committee member, ECE}: This committee provides the first round of evaluation of the tenure dossiers of tenure-track professors at the ECE Department. Accidentally, there were no candidates for tenure consideration in the given year, thus there were no specific duties on my side. 
\item[2016-17] \textbf{Graduate Student Financing Committee, ECE}: This committee ranks graduate students for awards and scholarships. My contributions included to rank NSERC Masters applications, NSERC doctoral applications, and Ph.D. students for the McGill Engineering Doctoral Award (MEDA) and to discuss the ranking with other committee members.
\end{yearlist}
%\end{description}

\subsection{International academic committees}
On an international level, I am an elected member of the Informatics Committee (with a total of 15 members) at the Hungarian Academy of Sciences, which committee is in charge of evaluting profiles of senior applicants to the Doctor of Science degree in Hungary (which is a formal prerequisite of promotion to full professorship in Hungary). I have been re-elected three consecutive times, last time 2017. I remained as a core member of the Informatics Doctoral School at the Budapest University of Technology and Economics (BME). Moreover, I served an external member of the Informatics Doctoral School at the University of Szeged until the end of 2016.


 
