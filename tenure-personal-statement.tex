\chapter{Personal Statement}
\label{sec:personal-statement}
\lfoot{Curriculum Vitae} 

After a successful academic career pursued in Hungary at the Budapest University of Technology and Economics where I got tenure in 2009 and promoted to a full professor in 2014, I joined the Department of Electrical and Computer Engineering at McGill University as a full professor in August 2016. Since 2015, I have held a research chair position in Hungary leaing a research program on cyber-physical systems as part of the prestigious and highly selective MTA Lendület program. 
This is an annually awarded in Hungary with only 12-15 researchers across all disciplines, and with only three past awardees in computer science since the foundation of the program in 2009. 

While this dossier dominantly presents details of my achievements after joining McGill University, I also provide a brief summaries about the pre-McGill period of my career. Furthermore, as a unique aspect of my research portfolio, I initiated and led an international research program involving more than 10 PhD students and several master's students both from Canada and Hungary.

\paragraph{Research.}
My main research area is \emph{model-based software and systems engineering} aiming for the design, analysis, optimization or deployment of traditional safety-critical systems as well as smart cyber-physical systems (CPS). My research contributed precise and scalable graph-based software techniques as foundations of design-time and run-time tools used in domains like avionics, automotive and various Internet-of-Things (IoT) systems. Since joining McGill University, we managed to substantially advance the state-of-the-art by (1) introducing distributed graph queries for runtime monitoring, (2) developing a novel family of automated model generator for domain-specific graph models, and (3) proposing various techniques secure collaborative modeling. Major ongoing multidisciplinary research carried out at McGill University aims to support the digital multidisciplinary analysis and design optimization platform.

I have been continuously aiming to publish my research results at top scientific conferences and journals of model-driven engineering (my direct research area) or alternatively, at general software engineering venues. In my career, I have published 182 papers (36 since joining McGill) including 47 papers in peer-reviewed journals (10 at McGill), and 73 peer-reviewed conference papers (20 at McGill). I have published several papers at a large variety of major scientific venues of my research area including IEEE/ACM sponsored conferences of software systems such as e.g. MODELS, ICSE, ASE, ESEC/FSE and top journals (e.g. Software and Systems Modeling, IEEE Software, Software Tools on Technology Transfer, Science of Computer Programming, Information and Software Technology). My papers received over 6710 citations (according to \href{https://scholar.google.ca/citations?user=4Ya6dVoAAAAJ&hl=en}{GoogleScholar}) with a Hirsch-index of 43. As of today, I am the most cited software engineering researcher at McGill University.  

As a recognization of our research, \emph{seven papers} co-authored by my graduate students and myself received \emph{Best / Distinguished Paper Awards} at leading conferences (4x at MODELS, 1x at ASE 2011, 1x at CSMR-WCRE 2014 and 1x ETAPS 2018). Three of these papers were published after joining McGill University. Furthermore, \emph{two 10-year Most Influential Paper Awards} for our pioneering work on generic and meta-transformations (awarded at IEEE/ACM MODELS 2014) and benchmarking graph transformation tools (recognized at IEEE VL/HCC 2016).

I gave a total of 35 invited talks (10 talks since joining McGill) at international graduate schools, research seminars, at various companies (e.g. Ericsson, Embraer, Rockwell Collins) and research institutes (e.g NASA Jet Propulsion Lab, SRI International) - 10 of these talks were delivered after joining McGill. I was invited to give a keynote talk at leading software engineering conferences (CSMR 2012 and SOFSEM 2016), and a plenary talk at MODELS 2014 upon receiving the Most Influential Paper award. Moreover, I gave keynote talks at national conferences (CSER 2017 in Canada and CSCS 2016 in Hungary) and at numerous workshops. I gave a total of 9 talks and 4 tutorials at international summer schools.% (DSM-TP, SERENE, SENSUS). 

I was successful in securing competitive research funding on a Canadian, European and Hungarian level by acquiring an equivalent of over 4.5 million CAD for my own research. This includes competitive funding over 600,000 CAD in Canada (after joining McGill) along projects in the NSERC Discovery Grant program and the Collaborative Research and Development program with Siemens. Prior to succeeding as a Lendület research chair at the Hungarian Academy of Sciences, my ERC Starting Grant project CERTIMOT (in 2009) went through to the final round, and it was recommended for funding, but become out of budget, but it still received partial national funding later. I acted as the site leader (or research director at BME) of five collaborative European projects.% in the FP6 and FP7 framework in the field of service-oriented computing (SENSORIA), avionics (DIANA), evolving secure systems (SecureChange), transport (E-Freight) and scalable model-driven engineering (MONDO). 
I was the PI of industrial projects with Embraer, Ericsson and Nokia. I am a three-time recipient of the IBM Faculty Award.

\paragraph{Student supervision.}
I have been the main supervisor of 11 PhD students and a co-supervisor for 6 PhD students with 11 successful defenses up to now. Seven of those PhD students had a McGill affiliation (as a PhD student or as a graduate research trainee at McGill). I have also supervised 5 MEng students and 38 undergraduate students and at McGill University (with career total of 28 MSc/MEng students). 115 of my research papers have at least one student co-author I supervised or co-supervised (with 32 out of 34 papers after joining McGill). My graduate students successfully competed in various doctoral research competitions organized at top international conferences achieving twice a 1st prize and once a 3rd prize after I joined McGill. 
%Previously, I was selected as (the youngest ever) Distinguished Tutor, a bi-annual national prize requiring 10 years of successful tutoring. 
In 2017, I was the first ever recipient of the Csanád Imreh Award where a single award is given bi-annually to scientists in the field of computer science below the age of 41. 

\paragraph{Service.}
Since joining McGill University, I was invited to serve in various senior organizational roles at major international conferences, such as the program co-chair (SLE 2016, MiSE 2019), posters co-chair (ICSE 2019), program board member (MODELS 2016 and 2017), steering committee member (ICMT) or editorial board member (SoSyM, JOT). I served on the program committee of major international conferences of my field (including e.g. ICSE, MODELS, ICMT, ECMFA). At ICSE 2018, I received an \emph{ACM Distinguished Reviewer} award for my thorough reviews. I served as an external reviewer for various international projects and grants in 4 countries. Previously, I served as the program co-chair of FASE 2013, ICMT 2014, general chair of AGTIVE 2011 and STAF 2013, and the local organizing chair of ETAPS 2008 and EDCC-5, and served on the program committee of over 100 conferences and workshops.

At McGill University, I have served on 8 different committees for the ECE Department and 2 faculty-level working groups. For example, as a member of the Search Committee of the ECE Department in 2018 and 2019, I evaluated application packages of hundreds of applicants. Since 2018, I have been an elected representative of the Faculty of Engineering for the Council of Graduate and Postgraduate Studies (CGPS). In addition, I have been serving as an internal or external examiner for 5 PhD theses after joining McGill (with a career total of 24 PhD thesis reviews) in four different countries.


\paragraph{Teaching.}
Within the five courses I taught at McGill University, my most significant achievements were in the ECSE 321 Introduction to Software Engineering course where I successfully modernized the software technologies, brought in real customers for the complex software engineering project, redeveloped tutorials, and initiated the development of an autograder software for objectively evaluating technology assignments of students. Students are very supportive of my teaching style as 90\% of my instructor-specific course evaluation scores are over 4.0 (out of 5.0). Finally, I also gave 3 invited lectures and 2 tutorials at the international DSM-TP summer school. I also serve as the program director of the upcoming software engineering co-op program (expected to start in 2020) where undergraduate students will need to spend four compulsory internship terms at companies. In that role, I have been working on the key policies, guidelines, student and employer evaluation forms related to the co-op terms since Fall 2018.

%Teaching experience and course development. Since 2003, I have been involved in the development of 10 university courses on different levels (for BSc, MSc and PhD students) first at BME and then at McGill University. I was in charge of developing an undergraduate specialization on Systems engineering now an accredited curriculum at BME. I have been lecturing regularly for 70-120 on different levels (and occasionally for 200+ undergraduates). At McGill, I am the program director for the upcoming software engineering co-op program where undergraduates spend compulsory internships at companies as part of their curriculum. I am a co-author of a (Hungarian) textbook and I was an invited panelist at MODELS 2009 Educators’ Symposium. 

\paragraph{Experience in software industry.}
In 2013, I co-founded IncQuery Labs, an innovative Hungarian company, together with the first generation of my former PhD students (who are current executives of the company). I played a major role in acquiring the first large-scale projects e.g. with Ericsson and TeqBall and participating in its first European project OpenCPS. In 2016, the company received a 3rd place on the Deloitte Rising Star list as the fastest growing Hungarian company in Central Europe. In 2018, I was representing the company as part of the SysML 2.0 standardization in the Object Management Group. 

As further industrial experience, I was the founder and (until 2016) the main strategist of the open source VIATRA model query and transformation framework (used at Thales, Ericsson, NASA JPL, Airbus ThyssenKrupp Presta, several automotive tool vendors and in open tools like Capella, Artop or Papyrus) and a co-founder of MASSIF, a Matlab Simulink Integration Framework for Eclipse (used e.g. at INTECS, CEA, MapleSoft MBSE).  These projects started as research prototypes funded by research projects, and they gradually evolved to an industrial tool now maintained by IncQuery Labs Ltd.

%Previously, I was Vice President of Research at OptXware Ltd, a Hungarian spin-off company. I participated in a national project supporting the development of product family of service dependability and optimization.  After developing an Eclipse based toolkit for AUTOSAR, the company was partially acquired by a large international automotive company. I was one of the company representatives during the acquisition talks.
