\appendix
\chapter{Course Evaluation Results for ECSE 321 (W18)}

This appendix reproduces the complete set of comments from the ECSE 321 course which I received right after I initiated the modernization of software technologies used within the course. The course is a required course in the Software Engineering and Computer Engineering curriculum, as well as in the Software Engineering Minor selected by electrical engineers and mechanical engineers. As such, in each semester, the course receives some 75-100 students with a very different background, but with very little knowledge of software technologies of industrial relevance. While most of the students have completed one or two programming courses, this is the first time that they have to work on a real (complex) software engineering project. 


Below, I present the students' feedback separately for each question, and I highlight how the course feedback improved after I carried out the modernization of software technologies.

\begin{table}[h]
\footnotesize
\begin{tabular}{@{}p{12cm}p{1.1cm}p{1.1cm}@{}}
\toprule
\textbf{Question} & \textbf{W17} & \textbf{W18} \\ \toprule
\textbf{Q1: Overall, this is an excellent course.} & \textbf{3.8} & \textbf{4.5} \\ %\midrule
\bottomrule
\end{tabular}
\end{table}

\begin{itemize}
\item Great course, material is very representative of real life problems 
\item Provides a great overview of the software engineering practice. 
\item Even though it was a fully packed course, the amount of things we learned is amazing and I really enjoyed taking this class with Prof Varro 
\item Love the changes to this course, you actually learn about technologies that are being used in the industry currently which makes this course so valuable 
\item This course is definitely alot different than all my other courses that I have taken. It for sure had the steepest learning curve. My level of knowledge of software is so much higher now than when I started this semester. It really opened my eyes to my strengths and weaknesses in this field, and how much I really enjoy it. I loved the fact that it gave me a better sense of understanding of what I was getting into. Although the steep learning curve and the "learning by doing" them of this course is great, it sometimes had its downsides. It would seem sometimes a little chaotic, when we would just be handed something to do/deal with without getting remotely enough information on it ( eg: assignment 1, Jenkins, etc..). I understand that these in themselves were some of the best parts about the course (and the most I learned new things in), but the approach could have definitely been improved, maybe take things a tad slower for us. And, as Prof. Varro would say, nevertheless, I can say that this course gave me my favorite experience out of all my classes at McGill. Thank you! 
\item Learned lots of new things but every subject was gone over way too quickly. For someone who's minoring in software and majoring in electrical with no prior software experience, this class was a nightmare. Every subject was taught to us as if we already knew the subject material and every piece of jargon. 
\item This class in an awesome class. Maybe software tools that we will eventually use in our career thus nice to see them during school year. Adapts to new technologies thus improve with time to use software that are used out on the market 
\item 
\item Learned so much! 
\item Theory covered is good, but the scope of the work compared to what is taught feels disproportionate. The flow of the assignments and project felt odd as well. The entire first assignment was to implement a small application using a ton of new technologies, and we only learn about them/how to apply them/proper testing methods etc. after this assignment is due. While the assignment was great to have in terms of reference material, wasn't constructive to learning as much as the actual project was. 
\item This course made me develop a passion for programming (although I am an electrical engineering student). It has been the most interesting and educational course I have taken. So much effort is put in this course from the detailed Hands-On tutorial document, to the project requirements supported by ‘real’ clients, and finally, the interesting games! 
\item Varro really makes this class seem a lot easier than it really is, learned a great deal of useful topics and the main idea of software engineering. 
\end{itemize}

\begin{table}[h]
\footnotesize
\begin{tabular}{@{}p{12cm}p{1.1cm}p{1.1cm}@{}}
\toprule
\textbf{Question} & \textbf{W17} & \textbf{W18} \\ \toprule
\textbf{Q2: Overall, I learned a great deal from this course.} & \textbf{4.2} & \textbf{4.5} \\ %\midrule
\bottomrule
\end{tabular}
\end{table}

\begin{itemize}
\item The most I've ever learned from a single course! 
\item Would like to add that a bit more time should be spent on learning Android rather than just the setup 
\item I learned in this course more than i learned in my 3 years at McGill! 
\item Gives enough details about the different activities that shape the software development process, as well as guidelines and patterns to follow. 
\item I learned more in terms of programing than software 
\item Did not have a strong base, and I feel like even though it is a lot for one course, it is great. 
\item Was challenging, but helped learn a lot. 
\item Learning how to create a web frontend and Android frontend and connecting them to the backend was very interesting and made for an enjoyable project. 
\item Feels like too much of the content from this course overlaps with ECSE 223. 
\item I learned over 3 new programming languages within 3 months, I was able to design a web and android applications. I had an idea of how programming is used in the real market. 
\item As other students have said, there is a great deal to learn in this course; however, by the same token, there may be too much to learn. 
\end{itemize}

\begin{table}[h]
\footnotesize
\begin{tabular}{@{}p{12cm}p{1.1cm}p{1.1cm}@{}}
\toprule
\textbf{Question} & \textbf{W17} & \textbf{W18} \\ \toprule
\textbf{Q3: Overall, this instructor is an excellent teacher.} & \textbf{4.3} & \textbf{4.4} \\ %\midrule
\bottomrule
\end{tabular}
\end{table}

\begin{itemize}
\item Professor Varro is by far one of the most helpful, dedicated professors at McGill. I have never seen a professor go out of his way to this extent to help their students. His slides and assignments are there to make sure you've learned everything you need to learn. Hope to take my future classes with him too! 

\item Daniel Varro has been one of the best teachers I have had at McGill. He uses effective learning methods that force the student to truly understand the material. Professor Varro is a strong believer that students should not be spoonfed solutions and that self-learning is an invaluable skill. This course was therefore very refreshing in this sense and these teaching methods made me feel I was truly understanding the material which I found very fulfilling. Professor Varro is also quite industry-orientated and has adapted the course to match the current tools/languages currently being used in industry. I believe this is very important, particularly in computer/software where the skills and trades are shifting at fast paces. I would say Professor Varro is one of the few professors I have had at McGill who seems to truly care about his students and goes above and beyond to help them learn. Having said that, the professor was a little ambitious when it came to what he expected of the students. Many of us have a full course load as well as part-time jobs/ extra-curricular activities. Also as a Mechanical Engineering student my only previous knowledge was COMP208 so the learning curve was steep. This is a 3-credit course and it is hard to expect students to put 20hr+/week on deliverables. To help lessen this workload I would suggest that the project not include Android. Also the assignments, although relevant and helpful, could be taken out as the project is already substantial. Finally, for the final project deliverable I would suggest that it be less focused on documentation but more on functionality. We have spent a long time documenting all our processes and it is very tedious to have to document everything again when we are focused on completing the actual software! To conclude, Professor Varro has made this a fantastic and rewarding course. 
Starting with basic computational skills in C, Fortran and Matlab I now feel confident as a beginner Full-Stack engineer. 

\item Great professor, explains material clearly and succinctly. 

\item Amazing teacher! It's really amazing how much he's there for his student! It would be my honour to work with you some day! 

\item Cares about his students. Pushes and encourages excellence. Knowledgeable about the course. 

\item Can barely speak a word of English. Stutters nonstop and does thinks course material is completely useless. Meanwhile his slides are unreadable and completely incomprehensible when studying. 

\item Very intelligent man, very helpful and patient with his students. 

\item Extremely dedicated and knowledgeable prof 

\item Good person, interested in the success of his students, but lectures are a little boring... mid tone, classes are not so engaging... difficult too to make this type of content interesting when they are often things that we learn was we actually do work and work in teams.

\item He really cares that you learn the material while also challenging you 

\item The professors seems to be very passionate about the material which makes his classes very engaging. He encourages asking questions and makes sure we absorbed all the challenging material. He puts a lot of effort ensuring that we got all the help we needed for the project and the assignment. 

\item Very approachable prof, no bad comments 
\end{itemize}

\begin{table}[h]
\footnotesize
\begin{tabular}{@{}p{12cm}p{1.1cm}p{1.1cm}@{}}
\toprule
\textbf{Question} & \textbf{W17} & \textbf{W18} \\ \toprule
\textbf{Q4: Overall, I learned a great deal from this instructor.} & \textbf{4.2} & \textbf{4.3} \\ %\midrule
\bottomrule
\end{tabular}
\end{table}

\begin{itemize}
\item One of my top 5 professors at McGill. 
\item Tutorials were very beneficial. 
\item Had to learn most of the material by myself 
\end{itemize}

\begin{table}[h]
\footnotesize
\begin{tabular}{@{}p{12cm}p{1.1cm}p{1.1cm}@{}}
\toprule
\textbf{Question} & \textbf{W17} & \textbf{W18} \\ \toprule
\textbf{Q5. The instructor was well organised in class and presented the material clearly.} & \textbf{4.1} & \textbf{4.2} \\ %\midrule
\bottomrule
\end{tabular}
\end{table}

\begin{itemize}
\item It will be better if more detailed examples can be provided, especially for modeling part and integration and system testing.
\item Would be nice if the nice class material could more explicitly cover the project material as a side by side aid. 
\item Lecture slides are very informative and clear. The material covered was synchronous with the projects and the assignment. 
\item I think that the class slides were a little ambiguous when studying them for evaluation like project deliverables, assignments, and the final exam. 
\end{itemize}

\begin{table}[h]
\footnotesize
\begin{tabular}{@{}p{12cm}p{1.1cm}p{1.1cm}@{}}
\toprule
\textbf{Question} & \textbf{W17} & \textbf{W18} \\ \toprule
\textbf{Q6. The instructor used effective teaching methods.} & \textbf{4.2} & \textbf{4.2} \\ %\midrule
\bottomrule
\end{tabular}
\end{table}

\begin{itemize}
\item He encouraged participation during the class, which made us understand and absorb the material better. He used revision games to make the material entertaining. He offered bonus marks for outstanding performance, which was very motivational! 

\item Would be very helpful if more examples/questions on class material were provided to learn as we progress. 

\item I like the inclusion of group activities; other times, I find the class a little dry. 

\item Slides are not strong slides. Even with them, there is no chance in succeeding this class. Must listen to what he says. Thus, I believe that for future classes, it would be best if he gave the lecture slides before class, because even though they are given, they are not a lot of information on it, or at least precise since most of it thought by Varro verbally. Thus, I still believe that people will attend class and it is easier to take notes as he goes since not a lot of words are provided on the slides. 

\item The slides were not incredibly helpful or detailed. A great majority of things I learned from this class was during the project, which I do not think is a good thing. However, it would have been good to learn more from the lectures. 

\item I believe that professor Varro should first clearly present what he is talking about and how and why it works then go into the code. As a student, I was very confused what he was talking about. Assumes we know too much. 

\item More examples of diagrams please! 

\item The slides are good while in class (lecture accompanies by many visuals) but it is hard to study them outside of class due to lack of context/text in a lot of slides. 
\end{itemize}

\begin{table}[h]
\footnotesize
\begin{tabular}{@{}p{12cm}p{1.1cm}p{1.1cm}@{}}
\toprule
\textbf{Question} & \textbf{W17} & \textbf{W18} \\ \toprule
\textbf{Q7. The instructor was responsive to students’ questions and concerns, given the class size.} & \textbf{4.5} & \textbf{4.7} \\ %\midrule
\bottomrule
\end{tabular}
\end{table}

\begin{itemize}
\item The professor is very supportive and answers all our questions. He stays answering the students after each lecture, during the office hours, through emails and on myCourses discussion board where there's more than 200 posts. Most of the students are form other majors (taking a minor in Software), and they took at most 2 programming courses. So, they asked a lot of questions and required a lot of help. The professor and the TAs were always there for us. And the result: by the end of the year, all groups have created a complicated well designed web page and Android app!! 
\item Always available to help students. Never seen a professor go out of his way to this extent to answer all emails, myCourses discussion board questions within minutes. 
\item Very responsive and very helpful. He replies to emails and questions within minutes! 
\item One of his better qualities. 
\item Always responds to emails and helpful in office hours 
\item E-mails were almost instantaneous in response. Feedback is detailed. Available after class and office hours every week for questions/concerns. 
\item He had one office hour during my class hours and it was very busy. 
\item Gives office hours. However the concept of it, of being like a conference is not so helpful I believe because its often really long before someone with a precise question gets answered. 
\item Always responded right away! thank you!! 
\end{itemize}

\begin{table}[h]
\footnotesize
\begin{tabular}{@{}p{12cm}p{1.1cm}p{1.1cm}@{}}
\toprule
\textbf{Question} & \textbf{W17} & \textbf{W18} \\ \toprule
\textbf{Q8. The instructor fostered an environment of mutual respect and engagement in learning.} & \textbf{4.7} & \textbf{4.8} \\ %\midrule
\bottomrule
\end{tabular}
\end{table}

\begin{itemize}
\item Yes indeed! Did not feel stupid to ask questions. Varro was very receptive to all questions. 
\item Use of clickers helps keep students engaged, even if they don't know the answers. 
\item The class is very diverse in terms of majors (since it's a minor course), so the professor assigned the first assignment in groups to ensure that we involve with other students. He also encouraged asking questions and provided us with resources to learn on our own. 
\end{itemize}

\begin{table}[h]
\footnotesize
\begin{tabular}{@{}p{12cm}p{1.1cm}p{1.1cm}@{}}
\toprule
\textbf{Question} & \textbf{W17} & \textbf{W18} \\ \toprule
\textbf{Q9. The course materials contributed to learning the subject matter.} & \textbf{4.2} & \textbf{4.5} \\ %\midrule
\bottomrule
\end{tabular}
\end{table}

\begin{itemize}
\item The first tutorial (Hands-On tutorial) was the most sufficient resource that we used throughout the final project. It was very detailed with screenshots and steps, as well as sources of error. It was very easy to follow especially with the buttons of the sections on the right. 
\item The project was very in-line with the course content. 
\item Had to learn a significant amount during office hours and by myself. 
\item Using a virtual machine was a terrible idea. I wasted many hours on this as it was extremely slow and it made completing the first assignment really really reslly much more complicated than it had to 
\end{itemize}

\begin{table}[h]
\footnotesize
\begin{tabular}{@{}p{12cm}p{1.1cm}p{1.1cm}@{}}
\toprule
\textbf{Question} & \textbf{W17} & \textbf{W18} \\ \toprule
\textbf{Q10. The course activities (inside and outside the classroom) engaged me actively in my learning process.} & \textbf{4.4} & \textbf{4.6} \\ %\midrule
\bottomrule
\end{tabular}
\end{table}

\begin{itemize}
\item The revision games provided with the classes were not only educational and engaging, but also helped form groups and involve with the class. Also, the examples given in the lecture were very helpful in understanding the material. Finally, the design of the assignment questions! Most teams divide the work for the project, so some members have experience in software while others are responsible for documentation. However, the assignment help ensure that each student has experience with the important software topics of this course. They were very helpful and the feedback on each assignment was very detailed. 

\item I am really happy that Varro updated the project of this course to reflect newer frameworks and marketable skills. It gives us, the students, a taste of many of the qualifications asked for in internship applications, such as popular frameworks, testing, automation, and development techniques. 


\item The project deliverables were very fun to work on and encouraged me to learn quite a bit of additional information. 

\item The deliverables and assignments taught me a lot. However i must admit this course was a lot of work: I think having deliverables and assignments is a lot, the deliverables take a lot of time! 

\item Too much I don't like engagement 

\item Have to go look for everything myself online. Course activities and assignments are meant to reflect things learned in the class. It's kind of hard to do that when everything is impossible to understand 

\item Building the app was probably the most enlightening activity I did in university. Learned a lot. The small assignments however were a waste of time. I would strongly sugest you get rid of the Virtual Machine next year. Gives us double the trouble. To figure out the new software plus the VM. You used it to make easier for you to help us but since 99\% of the time we are on our own you should not apply this idea.

\item Great Assignments to keep with the lectures, and the project help us learn a lot. 

\item The group project teaches you so much! 

\item Yeah!! Impossible to do the project without learning!! 

\item Being the first class to have new technologies tested on makes it more difficult to feel like we're actively learning but rather just struggling to figure out the new system being implemented. That being said, the professor and TAs operated very well with their new technologies and helped as much as possible. 

\item I think the mentors that we had this semester should try and reach out to groups that they are assigned too, I know that they may be busy undergrads just like us but similar to how ECSE 211 have their TAs set up meetings for their groups (I know that the TAs in 211 don't have nearly as many groups), but I think it would be an idea to look into if possible 

\end{itemize}

\begin{table}[h]
\footnotesize
\begin{tabular}{@{}p{12cm}p{1.1cm}p{1.1cm}@{}}
\toprule
\textbf{Question} & \textbf{W17} & \textbf{W18} \\ \toprule
\textbf{Q11. The evaluation methods used in this course were fair and appropriate.} & \textbf{4.2} & \textbf{4.3} \\ %\midrule
\bottomrule
\end{tabular}
\end{table}

\begin{itemize}
\item Grading scheme is fair. However, assignments and projects do not directly reflect the actual effort and understanding of students since they can use outside help for instance. There should be a more standardized evaluation such as a midterm. 

\item Group projects are terrible for people who are more competent than the average group member. You either have to do all the work to achieve your potential, or settle for a lower mark. 

\item None of the grading schemes were very descriptive in what they required. Got marks taken off because assignment instructions are way too vague. Project was supposed to be freeform but got many marks taken away for sections that involved more liberty. 

\item Varro was very fair with his expectations. 

\item I believe that the final exam was extremely difficult to complete because of how long it was. That is, I spent all three hours writing as fast as possible to ensure that I finished it. I also think that it was not extremely fair to have half of the exam be things from the lecture slides that we wrote on our crib sheet (this was just copying things from out crib sheet into our exam and didn't serve much purpose), and the other have of the exam require us to draw every possible diagram that we had seen, as making diagrams takes time and effort, and being rushed does not help us completing them. 

\item The one thing that irks me is the detail of some questions asked in this final as well as in-class ECSE 223 wherein some questions of vocabulary seem just pedantic, and by asking them while allowing open-book, these parts of the exam just become a test to see which student printed out the entire set of lecture notes to bring over the course of the semester for the sake of such a question. 
\end{itemize}

\begin{table}[h]
\footnotesize
\begin{tabular}{@{}p{12cm}p{1.1cm}p{1.1cm}@{}}
\toprule
\textbf{Question} & \textbf{W17} & \textbf{W18} \\ \toprule
\textbf{Q12. I was provided with useful feedback on my progress in the course.} & \textbf{3.8} & \textbf{4.4} \\ %\midrule
\bottomrule
\end{tabular}
\end{table}

\begin{itemize}
\item Since the project was cumulative, meaning that each deliverable depends on the previous one, it was very crucial to have clear feedback. Feedback was very fast on assignments, projects and the final exam. It was very detailed so we used it as a resource for later deliverable. And finally, the idea of bonuses was very motivational! 
\item Every single assignment was graded with very detailed feed-back so I knew what was correct, and what to improve. 
\item Really detailed correction. we know exactly where we made our mistakes, and if needed TAs are present to go over with us for more understanding. 
\item Very good feedback! 
\item Comments given on assignments are not useful at all. 
\item Feedback on one deliverable would often come after the next was already due, making it harder to identify and address issues as they appeared. 
\item The feedback was quick 
\item Feedback on assignments and deliverables, aswell as time spent in office hours, were very beneficial for my understanding on the course material. 
\item Varro was a speedy marker! Marked exams in 1 day!!!!!!!!! 
\end{itemize}

\begin{table}[h]
\footnotesize
\begin{tabular}{@{}p{12cm}p{1.1cm}p{1.1cm}@{}}
\toprule
\textbf{Question} & \textbf{W17} & \textbf{W18} \\ \toprule
\textbf{Q13. The course workload was appropriate, given the credit weight and the scheduled activity hours.} & \textbf{3.5} & \textbf{3.5} \\ %\midrule
\bottomrule
\end{tabular}
\end{table}

\begin{itemize}
\item It is a lot of work, but you do learn a lot 
\item The workload is a bit too much in terms of the hours needed to achieve the deliverable, but nonetheless I can't see how else we could learn from this course. You learn so much in this course, in the most interactive, engaging way. The project is so useful, and fundamental to this course. 
\item Workload for team project is large but can learn a lot from the project. 
\item Should be a 4 credit course with this amount of work. People are only just starting to learn This material and we can't be expected to be masters of it within a week. 
\item I really loved this course but I really think it's a lot of work, especially since most of us are taking 5 classes! 
\item Time spent on the project does not reflect the credit weight. Recommend to make the class 4 credits worth. 
\item I spend half my week working on this cources and the over half on my over 5 cources. Defently get rid of most of the assigments. 
\item We as a team have put way more hours into this class compared to others... could be a 4 credit class would be more appropriate considering all the new tools that one may not have learn before. The class was very overwhelming at first. But I guess it depends on the strength of people in programming thus a 3 credit class can also be appropriate. 
\item The project was very big, and took up a lot of time. I still think it was worth it though. 
\item Spent an incredible amount of time on this course(30+hours a week outside of class time), which is way more than the expected time I'd expect from a 3 credit course. Nevertheless, it was enjoyable. 
\item SO many hours of my life all weekends and evenings were dedicated to this course. The course SHOULD be worth 4 credits at least. Nevertheless I did enjoy the material. 
\item Scope of the project felt quite large given the time constraint and that many of the students in the class were taking ECSE 223 in parallel (and thus developing another application in their "down" time). 
\item This course should be a 4 credit course. 
\item I dont think the weight of each class component was fair in regards to the worklaod that came with each. The course project should be worth way way more as we spend the entire semester on it and for me even 40 hours a week on it. I didnt find it fair the final was worth so much and it was wayy too long of a final too. I couldnt even finish it. I think making it easier and worth less would be great. Also , I found assignments unnecessary. 
\item In teams where some individuals are not as experienced, the workload becomes heavily skewed. 
\end{itemize}

\begin{table}[h]
\footnotesize
\begin{tabular}{@{}p{12cm}p{1.1cm}p{1.1cm}@{}}
\toprule
\textbf{Question} & \textbf{W17} & \textbf{W18} \\ \toprule
\textbf{Q14. The course workload was appropriate, given the credit weight and the scheduled activity hours.} & \textbf{3.5} & \textbf{3.5} \\ %\midrule
\bottomrule
\end{tabular}
\end{table}

\begin{itemize}
\item This is by far the most relevant class for software engineering. Information learnt is practical and the patterns and technologies learnt in this class can be used in virtually every software project. 
\item I LOVE SOFTWARE ENGINEERING IT'S SO MUCH FUN. 
\item The first time I understood the significance of each and every word in the slide, because we were constantly applying it every day to tackle the projects and assignments. 
\item Documentation is boring 
\item Yeah appreciate the work software engineers do! Am glad to be in Computer engineering though, more high level. 
\item This course was very interesting and engaging. Therefore, I developed a passion for programming which made me decide to apply for a minor in Software engineering. 
\end{itemize}
